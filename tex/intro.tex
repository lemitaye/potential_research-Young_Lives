\section{Introduction}

In a multilinguistic society with historically favored and dominant local or international language, it is not clear whether the introduction of mother tongue instruction (MTI) leads to beneficial outcomes in economic development.  Learning in one's mother tongue can potentially enhance labor market outcomes if it improves human capital accumulation. Yet, the link between MTI and success in the labor market is not theoretically certain, nor empirically well-established, largely because the relationship depends on unobserved factors that could spur it in contradictory ways.  

While education is the key channel through which MTI could impact job market outcomes, it is not evident what the educational effects of MTI are, since the mediating factors \enquote{can either facilitate and optimize access to the content of the curriculum or block learning, preventing both access and equity} (Heugh et al, 2007). Different linguistic theories identify several variables that work in complex ways to influence the association between MTI and educational outcomes. Whereas sociolinguistic views highlight the relevance of relationships \enquote{between language and power} and the presence of significant variations in how  \enquote{linguistic communities make use of and manage the linguistic rights and resources at their disposal}; applied-linguistic theories focus on the significance of \enquote{heterogeneity of experiences in language teaching methodologies, design of language teaching programmes, and availability of textbooks and other learning materials} (Heugh et al, 2007). These views imply that the impact of choice of language of instruction on human capital development varies according to the social context in which it is implemented. 

A growing number of empirical studies have documented that mother tongue instruction in early schools leads to better learning outcomes (see, for instance, Walter and Chuo, 2012; Taylor and Von Fintel, 2016), with a few making the opposite case that the scheme results in reduction in schooling and literacy (e.g., Angrist and Lavy, 1997; Chicoine, 2019). Hence, learning in native languages can potentially increase participation in the labour market by raising educational attainment; however, gains in human capital associated with MTI may not necessarily lead to gains on the job market. For instance, people from less-favoured linguistic groups, who are schooled in their native languages, could face an undue penalty on the job market – notwithstanding their qualifications – because of market imperfections  which includes biases in favour of those with sufficient mastery of the dominant language. 

Indeed, while there are studies that show that early mother tongue education might help in mastering international languages better at a higher grade (e.g., Seid, 2019), several empirical studies show that MTI may hamper fluency in dominant languages, and this may in turn affect employment and earnings negatively (e.g., Chiswick and Miller, 2002). Further, in a country like Ethiopia, where the current study is based and there is a pre-existing dominant domestic language and a history of ethnic-based horizontal conflicts and discrimination (Henze, 1992)\footnote{Henze, Paul B., The defeat of the Derg and the establishment of new governments in Ethiopia and Eritrea. Santa Monica, CA: RAND Corporation, 1992. https://www.rand.org/pubs/papers/P7766.html. Also available in print form.},  the adoption of MTI by the speakers of less-favoured native languages can be a source of inbuilt biases and discrimination on the job market, thus adversely affecting the occupational and life outcomes of individuals obtaining their education in their native languages. 

Overall, there are two primary transmission mechanisms through which learning in non-dominant native languages affects employment and wage outcomes. The channels appear to generate contrasting effects – a positive effect through school performance, and a negative one by hampering proficiency in prominent languages and due to discrimination . Hence, whether mother-tongue instruction affects labour market outcomes is an empirical research question which calls for rigorous analysis in different contexts. In this paper, we strive to conduct a coherent empirical investigation on the subject, aiming to provide some answers to a research question that is still largely underexplored. 

Identification of the labour market impacts of MTI is a non-trivial undertaking, because the observed associations between MTI and labour market outcomes are influenced by a number of unobserved factors, in ways that are not straight-forward to accurately identify. To overcome the endogeneity of MTI in job market outcomes, we rely in this study on a major policy shift that introduced MTI in Ethiopia in 1994, following a regime change that installed ethnic federalism in the country. Prior to this change, Ethiopia had a dominant language, Amharic, which served as the official language of the Ethiopian state and the language of instruction in primary schools. 

The regime change created a number of ethnolinguistic regional states which – in accordance with the new national policy pertaining to languages of instructions in primary and middle schools – proceeded to adopt MTI in their schools \emph{ with different rates of intensity }\footnote{FN here regarding a map of Ethiopia divided into ethnic regions. The map should be included as an appendix.}. There are significant differences in how the different local states in the Ethiopian federation implemented the 1994 language of instruction policy, and it is this aspect of the scheme that we utilize to correctly gauge the economic impacts of MTI. Since the possible effects of mother tongue instruction on labour market outcomes are largely mediated through education, our initial task is to pin down the consequences of the policy on two different indicators of education, which we treat as intermediate outcomes. By so doing, we seek to contribute to the research on the likely payoffs of mother tongue instruction. 

The spatial variation in the take up of the policy at the state level generates differences in the distribution of native tongue instruction, and we conjecture that the disparity in the diffusion rates of MTI might have led to different rates of human capital accumulation and labour-market outcomes across the country. This presents a unique opportunity to test the research question using an IV-2SLS empirical framework, where the instrumental variable is defined based on an attribute of the students assigned by nature – their ethnicity.  To be more precise, the IV in this study is an index variable constructed using a unique interaction of student ethnicity and regional state policy re: MTI , thus it is exogenously determined in the empirical model. 

The evidence on the impact of MTI on labour market outcomes is mixed. While it increased the probability of salary employment by about 5\%, it did not seem to have any effect on wage employment. The conclusion that the introduction of MTI in Ethiopia has had a positive effect on the probability of gaining salaried jobs but no effect on wage employment is consistent with the prevailing social infrastructure in the country. We also find that mother tongue education enhances human capital accumulation; it improves student test scores in Mathematics and verbal comprehension.  

The introduction of MTI in Ethiopia has thus proven to be a double-edged sword for its beneficiaries.  Mother Tongue Instruction’s labour market advantages associated with the increase in human capital are limited only to salaried jobs which are mostly in the public sector, where employees have legal right to work in their native tongues. In the private sector, where mastery of the dominant language, Amharic, is still considered useful for gaining wage employment, the human capital benefits accruing from MTI appear to be offset by the unique disadvantages it imposes in terms of employability, perhaps due to its likely unintended adverse consequences on the mastery of the still somewhat dominant Amharic language.

To verify if the exclusion restriction implied by our IV-2SLS identification strategy holds, we conduct a test à la (Angrist, 1990), and rule out the existence of discernible correlations between the outcome variables (salary employment, wage employment and test scores) and the instrument in a sample where there is no obvious reason for association between the causal and the outcome variables. Additionally, we conduct a test that nullifies the possibility that there may be unobserved differences in state-level characteristics driving the results, providing another evidence corroborating our main findings.  

The rest of the paper proceeds as follows: Section two summarizes the empirical literature that is closely related to the current study. In section three, we describe the data as well as the institutional setting and the policy shock that made this study possible. After spelling out our key identification strategy in section four, we proceed to discuss the main findings in section five, contextualizing them with the social environment in which MTI was implemented. After demonstrating that our key findings remain robust by conducting some tests in section six, we conclude by examining the study’s prominent policy implication.
