
\section{Robustness Checks}

We check the veracity of the findings reported above by implementing two different procedures. First, we conduct a test to check whether the exclusion restriction implied by our identification strategy holds, by checking to see if there are correlations between the outcome variables and the instrument in a sample where there is no obvious reason for association between mother tongue instruction and the outcome variables in education and employment. As discussed earlier, our identification strategy would fail if the impact of ethnicity and intensity of mother tongue instruction affected human capital accumulation and job market outcomes in Addis Ababa, where we shouldn’t expect a meaningful association to exist between mother tongue instruction and the measured outcomes variables. 

Therefore, the confirmation in table 3 that the instrument has no discernible impact on human capital formation and labor market outcomes in Addis Ababa, satisfies the exclusion restriction and validates our identification strategy, lending credence to the main findings reported in table 2. We also report the IV estimates for the Addis Ababa sample in table 4. As expected, the intensity of mother tongue instruction doesn’t have statistically significant impacts on test scores and labor market outcomes in the capital, Addis Ababa.

As an additional test of robustness check, we employ an empirical strategy that allows us to compare the conditional educational and job market outcomes in states that have adopted the 1994 language policy \textit{similarly} --- designating one of the states as a benchmark case for comparison purposes --- to see if the observed variations in test scores and job market outcomes are caused by variations in unobserved state policies and institutions, not differences in the intensity of mother tongue instruction. The framework shown in (3), where all the variables are as defined above except $ T $, which now is modeled as an indicator variable that is switched on only for the benchmark state, can be used for this purpose, with $ \theta $ capturing the likely impacts, if any, of the unobserved state-level characteristics on educational and labor market outcomes. 

\begin{equation}
	Y_{is} = \alpha + \theta\cdot T + \beta\cdot I_{is} + X_{is}'\delta + \epsilon_{is}
\end{equation}

As noted above, only Oromia and Tigray implemented the 1994 language of instruction policy fully in our sample, with Tigray using the Sabean script (a pre-existing script that has been in use in the country for Millennia), while Oromia introduced a new writing system based on the Roman script that has attracted strong advocates and detractors (\textit{cite}). Consequently, regressions of (3) using the Young Livers Survey samples from these two states, with one of the regions as a benchmark state (we use Tigray as a benchmark case), may reveal if unobserved factors – including differences in the scripts used and other institutional and policy variables peculiar to each state – have contributed to differences in human capital accumulation and labor market outcomes. 

The estimates in table 4 suggest that unobserved differences in institutions and policies that were not accounted for in the empirical model do not appear to explain the observed spatial variation in human capital accumulation and job market outcomes in contemporary Ethiopia.  Although Tigray and Oromia, which implemented the 1994 language of instruction policy fully, can be differentiated from one another in terms of their institutional capacities and other policies they might have pursued since the early 1990’s (for instance, Tigray adopted the Sabean script for its language, while Oromia chose the Latin script), the conditional outcomes in education and employment in these states are fairly similar, providing another corroborating evidence for the positive effects of mother tongue instruction on cognitive capacity as well as employability in salaried positions/jobs. 

In addition to confirming the key findings reported in table 2, the similarity of the conditional educational and employment outcomes in Tigray and Oromia as reported in table 4, disproves the view that the choice of the Latin script by some states in the Ethiopian federation may have adversely affected student academic achievements (\textit{cite}). The revealing inference of these findings is that the measured outcomes in education and employment are neutral to the choice of script in the two states, suggesting that both states might have chosen the appropriate scripts for their respective languages or both scripts might be equally valid for both languages. 











