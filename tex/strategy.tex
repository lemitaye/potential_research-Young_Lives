
\section{Identification Strategy and Empirical Framework}

As described earlier, Oromia and Tigray implemented the 1994 language of instruction policy issued by the Ethiopian Ministry of Education fully, whereas the Amhara state and the Southern Nations and Nationalities Regional State (SNNR) implemented it only partially.\footnote{In this study, a regional state is said to have fully implemented the policy if students are provided the opportunity to study all subjects in their mother tongue in grades one through eight; states that implemented the policy only up to grade N where N is less than eight, are considered to have implemented the national policy partially.} Hence, the overwhelming majority of students attending schools in Oromia and Tigray (about 45\% of the country’s population live in the two states), take all subjects (except the Amharic language class) up to grade eight in their respective mother tongues. In the Amhara state, Amharic is the language of instruction for all subjects up to grade six, and all key subjects are taught in English beginning in grade seven, with the exception of the Awi and the Oromo zones.\footnote{The Amhara state allows students in the Awi and Oromo special administrative zones of the region to learn in their respective native tongues.} In the Southern Nations and Nationalities Regional state with about 18\% of the country’s population, mother tongue instruction is offered only up to grade four, and all subjects are taught in English starting in grade five. In the capital city --- Addis Ababa --- which is home to all of Ethiopia’s ethnic groups, Amharic is the language of instruction up to grade six, and all key subjects are taught in English beyond grade six.

We exploit these differences in how the language of instruction policy was implemented in the four regional states, to assess the impact of mother tongue education on educational and labor market outcomes. Based on a conjecture that the policy-induced disparity in the \emph{intensity of mother tongue instruction} (iMTI) across different states might have resulted in differences in human capital accumulation and labor-market outcomes, an IV-2SLS empirical strategy is implemented to explore the research question under consideration.   

Equation [1] is the key (2nd stage) regression equation, with the instrumental variable defined based on ethnicity as described below.




