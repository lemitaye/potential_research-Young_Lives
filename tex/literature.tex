
\section{Related Literature}

Empirical studies exploring the impact of mother tongue instruction on labour market outcomes are rare and their conclusions are markedly mixed. Generally, the effectiveness of a mother-tongue literacy program is highly sensitive to choices of inputs and outcome measures as well as implementation details (Kerwin and Thornton, 2020). Studies that have found that learning in native languages improves the employment and earnings of the beneficiaries link those gains to the positive effects of MTI on human capital accumulation (e.g., Eriksson, 2014; Seid, 2017). 

Despite the dearth of sufficient rigorous evidence demonstrating that gains in schooling due to MTI translate into gains in employment and earnings as might be expected, several studies show that mother tongue instruction has positive impacts on various measures of educational outcomes in different contexts (e.g., Alidou et al., 2006; Walter and Chuo, 2012; Eriksson, 2014; Taylor and von Fintel, 2016; Seid, 2016; Ramachandran, 2017; Laitin et al., 2019). In general, MTI is observed to increase participation in schools, and reduce grade repetition and dropouts (Benson, 2000, 2005; Bender et al., 2005). One channel for these results is better access: students could be more likely to enter school because they can understand the language. Moreover, mother tongue instruction in the formative years of education can raise the chances of building non-language cognitive skills such as literacy and numeracy (Eriksson, 2014; Trudell, 2012). 

On the other hand, there are studies that show that learning in a well-developed second language (usually international) instead of a local language improves outcomes (e.g., Angrist and Lavy, 1997; Munshi and Rosenzweig, 2006; Shastry, 2012; Casale and Posel, 2011; Parinduri and Org, 2018). Angrist and Lavy (1997) found that elimination of compulsory French instruction in Morocco led to a marked decline in French-language skills and reduced earnings among affected groups. More recently, Parinduri and Org (2018), using data from Malaysia, have shown that having English as a medium of instruction improves English proficiency and educational attainment, but has a weak link to later labour market outcomes. In a study conducted in Ethiopia, Chicoine (2019) asserts that the shift to mother tongue instruction has led to a reduction in schooling and had no impact on literacy, with the negative impact being concentrated in regions that made the switch from an Amharic script to Roman script. 

Where mother-tongue instruction may have resulted in less desirable labour market outcomes, reduced proficiency in a dominant national and international language appears to be the main culprit (e.g., Angrist and Lavy, 1997). In several studies, having greater proficiency in the dominant language is shown to be the key factor for success in the labour market (Chiswick and Repetto, 2000; Chiswick and Miller, 2002; Bleakley and Chin, 2004; Lang and Siniver, 2006; Aldashev et al., 2009). According to Kahn et al. (2019), being fluent in the dominant language enhances job seeking outcomes (through better access to information, for example), productivity on the job, and promotion to higher paying positions. Hence, early mother-tongue instruction, by hampering fluency in the dominant language, could negatively affect employment and other economic opportunities in the long-term, although it is unclear why there is a trade-off between learning in mother tongue and mastery of dominant languages. 

Most of the evidence on the effects of mother-tongue education is based on reforms or interventions comparing instruction in mother tongue education against an international (usually former colonial) language, such as English, French, Portuguese, etc. (e.g., Benson, 2000; Angrist and Lavy, 1997; Eriksson, 2014; Taylor and von Fintel, 2016; Laitin et al., 2019). Eriksson (2014) examined the effect of mother-tongue vs. English or Afrikaans instruction using the Bantu Education act of 1955 in South Africa as a natural experiment. She finds that increasing mother-tongue instruction for black students from four to six years positively affected wages, literacy, educational attainment, and English-speaking skills. Taylor and von Fintel (2016), also using data from South Africa, have found that mother tongue instruction in the early grades significantly improves English acquisition in later grades. 

Laitin et al. (2019) conducted a randomized evaluation of a local language schooling program in Cameroon and found that students that were exposed to three years of the local language (Kom) in early grades scored significantly higher than untreated students, who were instructed solely in English, in math and English tests. In contrast to these findings, Angrist and Lavy (1997) found that the shift in the language of instruction in Morocco from French to Arabic had negative effects on French writing skills and earnings. However, we should be careful to extrapolate such results to developing countries since the degree of exposure to the international language in day-to-day life is different in the two contexts (Ramachandran, 2017; Laitin et al., 2019).

In Ethiopia, the shift occurred from a dominant local language to a number of other ethnic languages. Amharic has been the only official language Ethiopia has had in all of its recorded history. And given that Ethiopia has never been officially colonized by European powers, there is an economic premium in favour of Amharic instead of a foreign language (e.g., English), unlike what is the case in most other African countries (Seid, 2019). One might rightly expect the behavioural responses of a shift in the language of instruction from a dominant local language to other ethnic languages to be different from those resulting from an international language to a domestic one. This is because a change in the language of instruction is not a politically neutral innovation. As Cummins (2009) rightly notes, “Use of a language as a medium of instruction confers recognition, status, and often economic benefits (e.g., teaching positions) on speakers of that language. \dots It is also a socio-political phenomenon that is implicated in the ongoing competition between social groups for material and symbolic resources.”
