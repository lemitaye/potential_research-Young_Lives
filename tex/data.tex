
\section{The Setting and Data}

Ethiopia is an ethnic federation of 83 different ethnolinguistic groups, where the three major groups – the Amhara, the Oromo, and the Tigray groups – constitute about 70 percent of the country’s population.\footnote{These numbers come from the country's most recent Census which was conducted in 2007.} The Amhara – which according to the most recent Census constitute slightly over 25 percent of the population – have played the dominant role politically until the early 1990’s, when the then government was militarily defeated and replaced by the so-called Ethiopian People’s Revolutionary Democratic Front (EPRDF) – a coalition of different ethnic-based liberation movements, in which the Tigray People’s Liberation Front (TPLF) played a significant role (Henze, 1992). 

Before this sea change took place in the country’s political landscape, Amharic – the language spoken by the Amhara people – was by far the most dominant language, serving as the official working language of the Ethiopian state as well as the language of learning in all primary schools (cite). With English being used as a medium of instruction for all subjects, Amharic was also taught as a subject in secondary schools in all corners of the country. The learning and dissemination of Amharic was enforced by state policy, and, for all practical reasons, proficiency in the language was essential for success in the job market.

With the ascension to power of the TPLF-led EPRDF government, the status of Amharic as the dominant language in the social and political life of the country diminished significantly. The EPRDF instituted ethnic federalism and issued a new language of instruction national policy in 1994, empowering the ethno-federal units that constitute the country to establish mother tongue instruction in primary and middle schools under their jurisdiction (MoE document). 

Interestingly, member states of the federation adopted the new language of instruction policy creatively and \emph{differently} to meet their specific needs, with some implementing it fully (up to and including the first eight grades of schooling, consistent with the policy), while others executed it only partially. The spatial variations in the take up of the policy at state level generate significant differences in the rate of students’ exposure to native language education, and the differences in the rates of diffusion of MTI are expected to have resulted in variations in human capital accumulation and labour market outcomes across the country. 

\subsection*{Data and Descriptive Figures}

We utilize the Young Lives Survey (YLS)\footnote{The YLS is a longitudinal survey of 12,000 children in four developing countries: Ethiopia, India (Andhra Pradesh and Telangana), Peru and Vietnam. It is administered by Young Lives, based at University of Oxford, Department of International Development.} conducted in Ethiopia for the study. The YLS is a longitudinal dataset in which a random sample of about 800 young people and their families – in four different regional states of Ethiopia (Amhara, Oromia, Southern Nations and Nationalities, and Tigray) and in the country’s capital – were followed and surveyed in five rounds every four years, beginning in 2001 when the subjects were eight-years old. A significant proportion of the surveyed subjects obtained their primary and middle school education in their mother tongue. The latest round of the YLS was conducted in 2016 when the subjects were 24, the majority of whom had completed their schooling and are in some form of employment or seeking employment. The middle rounds and the most recent round of the Young Lives Survey contain data on the outcome variables used in the current study – math and verbal scores (intermediate outcomes) and wage and salary employment (final outcomes). All four rounds contain rich sets of information pertaining to the characteristics of the subjects and their families.

Table 1 presents descriptive statistics for the variables used in the econometric analysis. The first three columns present summary statistics for the sample from the four regions included in the YLS, while the last three columns describe the sample from the capital, Addis Ababa. As will be explained later in some detail, mother tongue instruction is expected to be orthogonal to human capital accumulation and job market outcomes in Addis Ababa. We thus use the Addis Ababa sample for robustness check. The table clearly demonstrates that students in Addis Ababa have higher maths and language test scores compared to the rest of the country. Moreover, employment outcomes are better in Addis Ababa, which is in line with our expectation. Focusing on the non-Addis Ababa sample, we observe that about half have some sort of wage employment, whereas around 16\% are engaged in salaried employment by the fifth round. Consistent with the overall distribution of the Ethiopian population, approximately 73\% of the overall sample is constituted by rural households.

Figure 1 displays the percentage of pupils that were wage and salary employed by the last round. The region of Tigray outperforms the other regions in terms of both employment outcomes, closely being followed by Oromia. SNNRS and Amhara appear to have the lowest wage and salary employment rates, with the Amhara region and the SNNRS faring the worst in terms of salary employment and wage employment respectively. It is notable that the two regional states with higher rates of salary and wage employment, Tigray and Oromia, are those that implemented the 1994 language of instruction policy fully, whereas the underperformers – the Amhara and the SNNRS – implemented MTI in their curriculums only partially. Figure 2, on the other hand, is a kernel density estimate of standardized maths and language test scores in each region. Standardized language test score displays more variability across the regions, with students in Tigray and Amhara performing better than their counterparts in SNNP and Oromia. In contrast, the standardized maths test scores appear to be more stable.



