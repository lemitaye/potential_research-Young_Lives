
\section{Results and Interpretation}

\subsubsection*{Key Findings}

The key coefficient in \eqref{eq:01} is $ \beta $; it captures the impact of native tongue education on the intermediate outcomes (test scores in math and verbal reasoning) and the two final outcomes (the probabilities of salary and wage employments). Table 2 indicates that the probability of salary employment grew by about 5\%, while wage employment remained unresponsive to the nationwide introduction of mother tongue instruction. Interestingly, mother tongue education has enhanced human capital accumulation, improving test scores in Mathematics and verbal comprehension by average scores of 1.2 points 2 percentage points respectively. The conclusion that mother-tongue instruction has a positive effect on the probability of gaining salary employment is consistent with the nature of salary employment (types of salary-based jobs) in Ethiopia and the institutional and policy changes that have taken place in the country in the last three decades.  

A recent World Bank study shows that the public sector is the largest employer of non-farm labor in Ethiopia, followed by the service, manufacturing and the construction sectors.\footnote{For more detailed information on regarding the distribution of salary and wage employments in Ethiopia, please follow the following link.  \url{https://openknowledge.worldbank.org/bitstream/handle/10986/32093/Ethiopia-Employment-and-Jobs-Study.pdf?sequence=1&isAllowed=y}} Whereas the public sector is the largest employer of salaried jobs – with the education and health sectors leading the way – most service, manufacturing and construction jobs are private-sector wage employments.  Additionally, the bulk of public services are primarily delivered by the state governments, which use local languages to deliver these services. Whereas Amharic – the current working language of the federal government – had been the official working language throughout the country prior to the early 1990’s, the ethno-federal states in Ethiopia have now their own official working languages to go along with the changes made in the languages of instruction policy.  Indeed, public sector jobs at different levels of regional governments require potential applicants be conversant in the local language of the state. The finding that mother tongue instruction increases salary employment in formal employment where the public sector is the largest employer, is thus congruent with the inference that adopting native tongue instruction has a favorable effect on cognitive development, corroborating a reasonably well-established positive link between enhanced cognitive development and increased employment opportunities (\textit{cite}).

The insensitivity of wage employment to the introduction of mother tongue instruction (and the attendant increase in mathematical and verbal reasoning) is an outcome of the fact that most private sector jobs in Ethiopia are wage employments and being able to communicate in Amharic effectively is still considered an asset to obtaining and keeping these jobs. Regardless of which state they operate in, private employers do not face language requirements in recruitment and retention of their workers, and they prefer potential employees who are proficient in Amharic, which remains to be a key means of communication in urban centers throughout the country, if not the dominant language it once was. Hence, the majority of those who have benefited from the positive effects of mother tongue instruction in the country (about 75\% of the country’s population) might have lost a competitive edge in wage employments as a result of the policy.

To sum up, the introduction of mother tongue education in Ethiopia has proven to be a double-edged sword for its beneficiaries. Although it increased their competitiveness on the market for salaried jobs by raising their mathematical and verbal reasoning abilities, it has put them at a distinct disadvantage when it comes to securing private sector jobs, which require proficiency in Amharic – a language in which many might have become less conversant due to the policy changes in languages of instruction. It appears that Mother Tongue Instruction’s labor market advantages because of the associated increase in human capital are offset by the unique disadvantages it imposes in terms of employability in the private sector perhaps due to its likely unintended adverse consequences on the mastery of the still somewhat dominant Amharic language.



