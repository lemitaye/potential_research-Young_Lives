
\section{Conclusion and Policy Implications}

The key inference of the study is that mother tongue instruction improves employment outcomes by increasing cognitive capacity if it is complimented with conducive institutional and policy frameworks. In the absence of favorable social infrastructure that could support its effective implementation, MTI may not benefit the employment outcomes of the treated, though it may raise their schooling outcomes. The paper shows that MTI increases test scores and salary employment, but it has failed to yield a corresponding increase in wage employment. 

That MTI raises salary employment is very much an expected outcome of the key feature of the reconfiguration of the Ethiopian state along ethnic lines in the early 1990’s. The 1994 Ethiopian constitution created local governments whose official working languages are the various ethnic languages of the regional states, and these changes have increased the prospects of securing salaried positions by those that obtain their education in their native tongues. Indeed, the various functions of the state governments in Ethiopia are currently performed predominantly by those who have been trained in their native tongues. 

By the same token, MTI has failed to deliver in terms of wage employment, since the overall social infrastructure in Ethiopia does not support a more robust implementation of the language of instruction policy. Employees do not have legal rights to work in their native tongues in the private sector, and there are no laws in the books that prohibit private employers from requiring workers to speak the historically dominant language, Amharic. It appears that many private wage employers in urban centres throughout the country have continued to use Amharic in conducting their day-to-day operations, restricting the employment opportunities of those who are educated in their native tongues and may have lost some fluency in Amharic. 

The policy of mother tongue instruction that Ethiopia adopted in 1994 can thus be judged to have contributed positively to the country’s economic development, if only partially. The study also suggests that the impact of MTI on the country’s development can be further optimized by implementing creative social policies that support the major changes the country adopted in recent decades in terms of empowering the regional states to run their local affairs. It may be appropriate for regional states to look into instituting labour laws that afford workers the legal rights to work in their native languages, similar to what other multilingual countries (e.g., Canada, Switzerland, etc.) have successfully implemented. 

The laws can be instituted differently in different regional states based on the specific needs and capacities of each state, and they must always be weighed carefully against their potential adverse effect on economic efficiency due to restrictions on labour mobility within the country. For instance, where there are constraints of sufficiently trained local talent, the regional states can adopt laws that provide exceptions to employers in order to allow them to attract the requisite talent from the national labour market. However, while aiming to reduce or eliminate the loss of productivity embedded in the status quo due to a pre-existing bias in favour of Amharic – which as we have shown in this paper denies more qualified local talent from fully participating in the labour market – policy makers in the state governments should guard against undertaking measures that may introduce a different kind of inefficiency in their efforts to promote local languages.
